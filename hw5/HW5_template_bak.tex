\documentclass{article}
\usepackage[utf8]{inputenc}

\usepackage[utf8]{inputenc}
\usepackage{amsmath, amssymb, amsthm, enumerate, graphicx}
\usepackage[usenames,dvipsnames]{color}
\usepackage{bm}
\usepackage[colorlinks=true,urlcolor=blue]{hyperref}
\usepackage{geometry}
\geometry{margin=1in}
\usepackage{float}
\usepackage{graphics}
\setlength{\marginparwidth}{2.15cm}
\usepackage{booktabs}
\usepackage{enumitem}
\usepackage{epsfig}
\usepackage{setspace}
\usepackage{parskip}
\usepackage[normalem]{ulem}
\usepackage{tikz}
\usetikzlibrary{positioning}
\usepackage{pgfplots}
\usepackage[font=scriptsize]{subcaption}
\usepackage{float}
\usepackage[]{algorithm2e}
\usepackage{environ}
\usepackage{bbm}
\usepackage{graphicx}
\usepackage{titling}
\usepackage{url}
\usepackage{xcolor}
\usepackage{lipsum}
\usepackage{lastpage}
\usepackage[colorlinks=true,urlcolor=blue]{hyperref}
\usepackage{multicol}
\usepackage{tabularx}
\usepackage{url}
\usepackage[nottoc]{tocbibind}



\begin{document}

\section*{}
\begin{center}
  \centerline{\textsc{\LARGE Homework 5}}
  \vspace{0.5em}
  \centerline{\textsc{\Large Researching Applications of Machine Learning}}
  \vspace{1em}
  \textsc{\large CMU 10-601: Machine Learning (Spring 2017)} \\
  %% You can comment out these parts if you want.
  \url{https://piazza.com/cmu/spring2017/10601}
  \centerline{OUT: March 08, 2017}
  \centerline{SUMMARY DUE: March 22, 2017 11:59 PM}
  \centerline{PEER REVIEW DUE: April 5, 2017 11:59 PM} 
  \centerline{TAs: Rui Sun, Prakruthi Prabhakar, Simon Du, Sarah Schultz}
  \centerline{\textbf{\Large Please do not include any personally identification information}}
   \centerline{\textbf{\Large No name or ID}}
   %%
\end{center}



\section*{}
\begin{center}
\centerline{\textbf{\Large The title of the paper here}}
\end{center}
\section{Data Description}
\paragraph{I will use the "MovieLens 1M dataset", which comes from University of Minnesota researchers started from 1997(GroupLen research project). The MovieLens dataset is composed from 3 different .dat file. I think the challenges on this dataset is user bias For user bias part, the MovieLens data set only collect data from users with at least 20 ratings, however, for people who didn't like to rate movies, those people in the MovieLens dataset maybe no reference value for them. For timestamp part, the timestamp of "MovieLens" only represents the time users filled the data rather than the time user watched this movie, which maybe inaccurate sometimes.
}
If you want to cite some papers.~\cite{latexcompanion}
\section{Task Description}
\paragraph{The author introduced two implementation ways of Neighborhood-based collaborative filtering algorithm. One is user-based, another is item-based. The author detailed described how to choose the top-k users/items for user-based/item-based collaborative filtering, and compare the advantage/disadvantage of these two ways as well as offline and online efficiency. For me, I think there are two challenging things. One is space demand. Although this neighborhood-based algorithm leverage the offline preprocessed data to enhance the online efficiency, however, offline phase costs lots of space and time. Another challenge is sparsity issue, if the top-k nearest neighbors didn't rate move "X", user will not get recommendation of this movie.  
}
\paragraph{wwwwwwww}
\section{Method}
\section{Results}
\section{Questions about the Paper}


\begin{thebibliography}{1}
\bibitem{latexcompanion} 
Michel Goossens, Frank Mittelbach, and Alexander Samarin. 
\textit{The \LaTeX\ Companion}. 
Addison-Wesley, Reading, Massachusetts, 1993.
 
\end{thebibliography}


\end{document}
